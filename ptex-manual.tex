%#!make ptex-manual.pdf
\documentclass[a4paper,11pt]{jsarticle}
\usepackage[textwidth=42zw,lines=40,truedimen,centering]{geometry}
%\usepackage{amsmath,mathtools,amssymb,multicol,comment}
\usepackage{array}\usepackage[arrow]{xy}
\usepackage[dvipdfmx]{graphicx}
\usepackage[T1]{fontenc}
\usepackage{booktabs,enumitem,multicol}
\usepackage[defaultsups]{newpxtext}
\usepackage[zerostyle=c,straightquotes]{newtxtt}
\usepackage{newpxmath}
\usepackage[dvipdfmx]{hyperref}
\usepackage{pxjahyper}
\usepackage[bold,deluxe]{otf}
\usepackage{hologo}
\usepackage{makeidx}\makeindex

%%%%%%%%%%%%%%%%
% preamble copied from eptexdoc.tex
\makeatletter
\def\epTeX{$\varepsilon$-\pTeX}\def\eTeX{$\varepsilon$-\TeX}
\def\upTeX{u\pTeX}\def\pTeX{p\kern-.15em\TeX}
\def\headfont{\normalfont\mathversion{bold}\sffamily}
\def\.#1{\texorpdfstring{%
     \leavevmode\hbox{\texttt{\textbackslash#1}}\ifmmode\else\textcompwordmark\fi}{\textbackslash #1}}
\let\orig@cs=\.
\def\ind@cs#1{\orig@cs{#1}\indcs{#1}}
\def\indcs#1{\index{{\texttt{\textbackslash #1}}}}
\newlist{cslist}{description}1
\setlist[cslist]{%
  itemsep=\medskipamount,listparindent=1zw,
  font=\normalfont\mdseries, leftmargin=2zw,
  before=\advance\@listdepth\@ne,after=\advance\@listdepth\m@ne
}
\def\csitem[#1]{\item[\llap{$\blacktriangleright$~}\let\.=\ind@cs#1]\leavevmode\par}
\def\emph#1{{\bfseries\sffamily\gtfamily\mathversion{bold}#1}}
\setlist{leftmargin=2zw}

\catcode`\<=13
\xspcode`\\=3
\xspcode`\*=3
\xspcode`\-=3
\xspcode23=3 % \textcompwordmark
\def<#1>{\ensuremath{\langle\hbox{\normalfont\textit{#1}}\rangle}}

\font\man=manfnt at 10pt
\def\dbend{\raise0pt\hbox{\man\char'177}}
\AtBeginDvi{\special{pdf:mapfile otf-ipaex.map}}

\normalsize
\bigskipamount=\baselineskip
\medskipamount=.5\baselineskip
\smallskipamount=.25\baselineskip

\usepackage{etoolbox}
\patchcmd\@verbatim\@totalleftmargin{\dimexpr\@totalleftmargin+2zw\relax}{}{}

\newenvironment{dangerous}{%
  \ifnum\@listdepth>\z@
    \GenericError{}{Do not use `dangerous' environment inside any list}{}{}
  \fi
  \par\medskip
  \@tempdima=\dimexpr\textwidth-2zw\relax\small
  \divide\@tempdima by\dimexpr1zw\relax\@tempcnta=\@tempdima
  \leftskip=\dimexpr\textwidth-\@tempcnta zw\relax
    \@totalleftmargin\dimexpr\leftskip+0zw
    \linewidth=\dimexpr\@tempcnta zw-0zw
  \parindent1zw\noindent\kern-\leftskip\hbox to\leftskip{\dbend\hss}%
  \everypar{\everypar{}}\ignorespaces
}{\par\medskip}

\renewenvironment{theindex}{%
    \def\presectionname{}\def\postsectionname{}%
    \section*{\indexname}
    \@mkboth{\indexname}{\indexname}%
    \plainifnotempty % \thispagestyle{plain}
    \parindent\z@
    \parskip\z@ \@plus .3\p@\relax
    \let\item\@idxitem
    \raggedright
    \begin{multicols}{2}
  }{
    \end{multicols}
    \clearpage
  }

\def\tableautorefname{表}
\def\figureautorefname{図}
\def\HyRef@autoref#1#2{%
  \begingroup
    \Hy@safe@activestrue
    \expandafter\HyRef@autosetref\csname r@#2\endcsname{#2}{#1}%
  \endgroup\textcompwordmark %欧文ゴースト
}

%%%%%%%% macros for index (from doc.sty)
\newif\ifscan@allowed
\def\pfill{\unskip~\dotfill\penalty500\strut\nobreak
	  \dotfill~\nobreak}

\makeatother
%%%%%%%%%%%%%%%%

\def\_{\leavevmode\vrule width .45em height -.2ex depth .3ex\relax}
\frenchspacing
\begin{document}
\catcode`\<=13
\title{\emph{\pTeX マニュアル}}
\author{日本語\TeX 開発コミュニティ\null
\thanks{\url{https://texjp.org},\ 
e-mail: \texttt{issue(at)texjp.org}}}
\date{version p3.7.1,\today}
\maketitle

\tableofcontents

\newpage

% jtex.tex の「ラインブレーク」を簡略化
\section{禁則}

欧文と和文の処理で見かけ上最も大きな違いは,行分割処理であろう.
\begin{itemize}
\item 欧文中での行分割は,ハイフネーション処理等の
      特別な場合を除いて,単語中すなわち連続する文字列中は
      ブレークポイントとして選択されない.
\item 和文中では禁則(行頭禁則と行末禁則)の例外を除いて,
      全ての文字間がブレークポイントになり得る.
\end{itemize}
しかし,\pTeX の行分割は\TeX 内部の処理からさほど大きな変更を
加えられてはいない.というのも,\TeX の\emph{ペナルティ}\footnote{%
ペナルティとは,行分割時やページ分割時に
「その箇所がブレークポイントとしてどの程度適切であるか」を示す
一般的な評価値である(適切であれば負値,不適切であれば正値
とする).}という概念を和文の禁則処理にも適用しているためである.

行頭禁則と行末禁則を実現する方法として,\pTeX では「禁則テーブル」が
用意されている.このテーブルには
\begin{itemize}
\item 禁則文字
\item その文字に対応するペナルティ値
\item ペナルティの挿入位置
      (その文字の前に挿入するか後に挿入するかの別)
\end{itemize}
を登録でき,\pTeX は文章を読み込むたびにその文字が
テーブルに登録されているかどうかを調べ,登録されていれば
そのペナルティを文字の前後適切な位置に挿入する.

禁則テーブルに情報を登録する手段として,
以下のプリミティブが追加されている.

% アスキー公式サイトのプリミティブ一覧から適当にまとめ
\begin{cslist}
\csitem[\.{prebreakpenalty} <16-bit number>=<number>]
  指定した文字の前方にペナルティを挿入する.
  正の値を与えると行頭禁則の指定にあたる.
  例えば\verb|\prebreakpenalty`。=10000|とすれば,句点の
  直前に10000のペナルティが付けられ,行頭禁則文字の対象となる.

\csitem[\.{postbreakpenalty} <16-bit number>=<number>]
  指定した文字の後方にペナルティを挿入する.
  正の値を与えると行末禁則の指定にあたる.
  例えば\verb|\postbreakpenalty`(=10000|とすれば,始め丸括弧の
  直後に10000のペナルティが付けられ,行末禁則文字の対象となる.
\end{cslist}

全角文字,半角文字の区別無しに指定できる.また,同一の文字に対して
両方のペナルティを同時に与えるような指定はできない
(もし両方指定された場合,後から指定されたものに置き換えられる).
``\verb|{}|''によるグルーピング規則が適用される.
指定できる文字数は最大で256文字までである.

この禁則テーブルからの登録の削除は
ペナルティ値0を設定することによって行われる.
グローバルレベルでの設定でなければ,このテーブルの領域は
解放されないことに注意.

% [TODO] ハッシュ関数の説明は jtex.tex にあるが,転記するか?
%% 別になくてもいいんじゃないんですか?(北川)

\begin{cslist}
\csitem[\.{jcharwidowpenalty}=<number>]
  パラグラフの最終行が「す。」のように孤立するのを防ぐための
  ペナルティを設定する.
  % このペナルティは,
  % 「段落が和文文字の1つ以上の連続%
  % \footnote{%
  %   正確には,和文文字を1つ以上含むような
  %   「和文文字,ペナルティ,グルー,カーン,文中数式の境界,\textit{mark\_node},
  %   \textit{adjust\_node}, \textit{whatsit\_node}, \textit{disp\_node}」の1つ以上の連続.
  %   ただし,アクセント本体は無視される.
  % }で終わる場合,
  % その中の最後にある\.{kcatcode}が16(漢字)または17(仮名)の文字%
  % \footnote{%
  %   \pTeX の初期値では,1,~2,~7--8区\emph{以外}に含まれる文字.
  % }の直前」に挿入される.
  % % 警告:現行の pTeX では,この仕様に沿っていません(欧文が絡んだ場合).
  % % 仕様を明確に決めるとしたらこんな感じになるしょうか.
  % %
  % また,その位置がすでにペナルティであった場合は,\.{jcharwidowpenalty}の値が合算される.
\end{cslist}

\section{文字間のスペース}

% [TODO] 禁則処理だけでは美しい行分割ができないのでスペーシングの
% 調節が必要,という流れとするとよさそう?

\begin{cslist}
\csitem[\.{kanjiskip}=<skip>]
連続する和文文字間に標準で入るグルーを設定する.段落途中でこの値を変えても影響はなく,
段落終了時の値が段落全体にわたって用いられる.
\end{cslist}

\begin{dangerous}
 和文文字を表すノードが連続した場合,その間に\.{kanjiskip}があるものとして行分割やボック
 スの寸法計算が行われる.\.{kanjiskip}の大部分はこのように暗黙のうちに挿入されるものである
 ので,\.{lastskip}などで取得することはできない.

 その一方で,ノードの形で明示的に挿入される\.{kanjiskip}も存在する.このようになるのは次の
 場合である:
 \begin{itemize}
  \item 水平ボックス(\.{hbox})が和文文字で開始しており,そのボックスの直前が和
	文文字であった場合,ボックスの直前に\.{kanjiskip}が挿入される.
  \item 水平ボックス(\.{hbox})が和文文字で終了しており,そのボックスの直後が和
	文文字であった場合,ボックスの直後に\.{kanjiskip}が挿入される.
  \item 連続した和文文字の間にペナルティがあった場合,暗黙の\.{kanjiskip}が
	挿入されないので明示的にノードが作られる.
 \end{itemize}
  しばしば
  
  \medskip\noindent
  \begin{minipage}{.6\linewidth}
	\begin{verbatim}
	\hbox to 15zw{%
	  \kanjiskip=0pt plus 1fil 
	  あ「い」うえ,お}
	\end{verbatim}
  \end{minipage}
  \begin{minipage}{.3\linewidth}
	\hbox to 15zw{%
	  \kanjiskip=0pt plus 1fil 
	  あ「い」うえ,お}
  \end{minipage}\medskip
  
  \noindent
  のように\.{kanjiskip}に無限の伸長度を持たせることで均等割付を行おうとするコードを見かける
  が,連続する和文文字の間にはメトリック由来の空白と\.{kanjiskip}は同時には入らないので,
  上に書いたコードは不適切である.
\end{dangerous}

\begin{cslist}
\csitem[\.{xkanjiskip}]
\end{cslist}


\.{xspcode}
\.{inhibitxspcode}

\.{autospacing}
\.{noautospacing}
\.{autoxspacing}
\.{noautoxspacing}
\.{showmode}

\.{inhibitglue}

\section{組方向}

% ptexdoc.tex の「ディレクション」を簡略化
従来の\TeX では,字送り方向が水平右向き(→),行送り方向が垂直下向き(↓)
に固定されていた。\pTeX では,\TeX の状態として“組方向(ディレクション)”
を考え,ディレクションによって字送り方向と行送り方向を変えることにしてある.
なお,行は水平ボックス(horizontal box),ページは垂直ボックス(vertical box)である
という点は,\pTeX でも従来の\TeX と同様である.

\pTeX のサポートする組方向は\emph{横組},\emph{縦組},そして
\emph{DtoU方向}\footnote{下から上(Down to Up)であろう.}の3つである.
また数式モード中で作られたボックスは数式ディレクションというまた別の状態になる.
横組での数式ディレクション(横数式ディレクション)と
\begin{dangerous}
あああああああああああああああああああああああああああああ
あああああああああああああああああああああああああああああ
あああああああああああああああああああああああああああああ
あああああああああああああああああああああああああああああ
あああああああああああああああああああああああああああああ
\begin{itemize}
 \item いいいいいいいいいいいいいいいいいいいいいいいいいい
いいいいいいいいいいいいいいいいいいいいいい
\end{itemize}
\end{dangerous}
DtoU方向での数式ディレクションはそれぞれ非数式の場合と区別はないが,
縦組での数式ディレクション(\emph{縦数式ディレクション})では横組を時計回りに
90度回転させたような状態となる.
従って,\pTeX では縦数式ディレクションまで含めると合計4種類の組方向がサポートされていると
いえる(\autoref{tab-dir}).

\begin{table}[tbp]
\caption{\pTeX のサポートする組方向}
\label{tab-dir}
\centering\small
\def\mathmode{}
\def\obox#1{%
  \setbox0=\hbox{\yoko\hbox{#1%
  \large\tbaselineshift0pt%
  \vrule height 25pt width 0.4pt depth 15pt\kern-.2pt%
  \raise25pt\hbox to 0pt{\hss\composite{*r^@{>>}}\hss}%
  \raise-15pt\hbox to 0pt{\hss\composite{*l^@{>|}}\hss}\kern.2pt%
  \vrule height.2pt depth.2pt width 60pt\hbox to 0pt{\hss\composite{*d^@{>}}\hss}\kern-60pt
  \mathmode\hbox to 60pt{\,銀は、Ag\hss}\mathmode}}%
  \raise\dimexpr 0.5\dp0-0.5\ht0\box0%
}
\begin{tabular}{>{\bfseries}lcccc}
\toprule
&横組&縦組&DtoU方向&縦数式ディレクション\\
\midrule
命令&\.{yoko}&\.{tate}&\.{dtou}&---\\
字送り方向&水平右向き(→)&垂直下向き(↓)&垂直上向き(↑)&垂直下向き(↓)\\
行送り方向&垂直下向き(↓)&水平左向き(←)&水平右向き(→)&水平左向き(←)\\
\parbox[t]{6zw}{使用する和文フォント}&横組用(\.{jfont})&縦組用(\.{tfont})&
\multicolumn{2}{c}{%
  横組用(\.{jfont})の$90^\circ$回転}\\[\smallskipamount]
  組版例 &\obox{\yoko}&\obox{\tate}&\obox{\dtou}&\def\mathmode{$}\obox{\tate}\\
\noalign{\medskip}
\bottomrule
\end{tabular}
\end{table}


以下が,組方向の変更や現在の組方向判定に関わるプリミティブの一覧である.

% アスキー公式サイトのプリミティブ一覧から適当にまとめ
\begin{cslist}
\csitem[\.{tate}, \.{yoko}, \.{dtou}]
  組方向をそれぞれ縦組,横組,DtoU方向に変更する.
  カレントの和文フォントは縦組では縦組用フォント,
  横組およびDtoU方向では横組用フォントになる.

  ディレクションの変更は,作成中のリストやボックスに何も入力されていない
  状態でのみ許される.違反した場合,
\begin{verbatim}
! Use `\tate' at top of list.
\end{verbatim}
  のようなエラーが出る.

  また,ボックスを\.{copy}したり中身を\.{unhbox}や\.{unvbox}で取り出し
  たりする場合は,同じ組方向のボックス内でなければならない.
  違反した場合,
\begin{verbatim}
! Incompatible direction list can't be unboxed.
\end{verbatim}
  なるエラーが出る.

\csitem[\.{iftdir}, \.{ifydir}, \.{ifddir}, \.{ifmdir}]
  現在の組方向を判定する.\.{iftdir}, \.{ifydir}, \.{ifddir}はそれぞれ
  縦組,横組,DtoU方向であるかどうかを判定する(数式ディレクションであるかは問わない).
  一方,\.{ifmdir}は数式ディレクションであるかどうかを判定する.

  従って,\autoref{tab-dir}に示した4つの状況のどれに属するかは以下のようにして判定できることになる.
  \begin{verbatim}
\iftdir
  \ifmdir
    (縦数式ディレクション)
  \else
    (通常の縦組)
  \fi
\else\ifydir
  (横組)
\else
  (DtoU方向)
\fi
  \end{verbatim}
  
\csitem[\.{iftbox} <number>, \.{ifybox} <number>, \.{ifdbox} <number>, \.{ifmbox} <number>]
  <number>番のボックスの組方向を判定する.<number>は有効なboxレジスタでなければならない.

  バージョンp3.7までの\pTeX では,ボックスが一旦ノードとして組まれてしまうと,
  通常の縦組で組まれているのか,それとも縦数式ディレクションで組まれているのか
  という情報が失われていた.しかし,それでは後述の「ベースライン補正の戻し量」を
  誤り,欧文の垂直位置が揃わないという問題が生じた(\cite{tatemath}).
  この問題を解決する副産物として,バージョンp3.7.1で\.{ifmbox}プリミティブが実装された.
\end{cslist}

\section{ベースライン補正}

\.{tbaselineshift}
\.{ybaselineshift}

\.{textbaselineshiftfactor}
\.{scriptbaselineshiftfactor}
\.{scriptscriptbaselineshiftfactor}

\section{和文ファミリ}

\.{jfam}
\.{jfont}
\.{tfont}

\.{kcatcode}

\section{文字コード変換,漢数字}

\pTeX の内部コードは環境によって異なる.そこで,異なる文字コード間で同じ文字を
表現するため,文字コードから内部コードへの変換プリミティブが用意されている.
また,漢数字を出力するプリミティブも用意されている\footnote{%
実は\.{kansuji}, \.{kansujichar}プリミティブはp3.1.1でいったん削除され,
p3.1.2で復活したという経緯がある.}.

% jtexdoc.tex の「追加されたプリミティブ」を簡略化
\begin{cslist}
\csitem[\.{kuten} <16-bit number>]
  区点コードから内部コードへの変換を行う.
  16進4桁の上2桁が区,下2桁が点であると解釈する.
  たとえば,\verb|\char\kuten"253C|は,「\char\kuten"253C」(37区60点)である.

\csitem[\.{jis} <16-bit number>, \.{euc} <16-bit number>, \.{sjis} <16-bit number>]
  それぞれJISコード,EUCコード,Shift\_JISコードから内部コードへの変換を行う.
  たとえば,\verb|\char\jis"346E|,\verb|\char\euc"B0A5|,\verb|\char\sjis"8A79|は,
  それぞれ「\char\jis"346E」,「\char\euc"B0A5」,「\char\sjis"8A79」である.

\csitem[\.{kansuji} <number>, \.{kansujichar} <0--9>=<16-bit number>]
  \.{kansuji}は,続く数値<number>を漢数字の文字列で出力する.
  出力される文字は\.{kansujichar}で指定できる(デフォルトは「〇一二三四五六七八九」).
  たとえば
\begin{verbatim}
\kansuji 1978年
\end{verbatim}
  は「\kansuji 1978年」と出力され,
\begin{verbatim}
\kansujichar1=`壱
\kansujichar2=\euc"C6F5\relax
\kansujichar3=\jis"3B32\relax
\kansuji 1234
\end{verbatim}
  は「{\kansujichar1=`壱 \kansujichar2=\euc"C6F5\relax
       \kansujichar3=\jis"3B32\relax \kansuji 1234}」と出力される.
\end{cslist}

% luatexja.dtx より pTeX 用に移植
以上に挙げたプリミティブ(\.{kuten}, \.{jis}, \.{euc}, \.{sjis}, \.{kansuji})は
内部整数を引数にとるが,実行結果は\emph{文字列}であることに注意.
また,\pTeX はJIS X 0213には対応せず,JIS X 0208の範囲のみ扱える.

\medskip\noindent
\begin{minipage}{.6\linewidth}
\begin{verbatim}
    \newcount\hoge
    \hoge="2423
    \the\hoge, \kansuji\hoge\\
    \jis\hoge, \char\jis\hoge\\
    \kansuji1701
\end{verbatim}
\end{minipage}
\begin{minipage}{.3\linewidth}
    \newcount\hoge
    \hoge="2423 
    \the\hoge, \kansuji\hoge\\
    \jis\hoge, \char\jis\hoge\\
    \kansuji1701
\end{minipage}\medskip

\section{長さ単位}

zw
zh
Q
H

\begin{thebibliography}{99}
 \bibitem{tatemath} aminophen, 「縦数式ディレクションとベースライン補正」,
  2016/09/05,\\
  \url{https://github.com/texjporg/platex/issues/22}
\end{thebibliography}

\newpage
\printindex


\end{document}

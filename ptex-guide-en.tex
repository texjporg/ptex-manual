%#!make ptex-guide-en.pdf
\documentclass[a4paper,11pt,dvipdfmx]{article}
\usepackage[textwidth=42zw,lines=40,truedimen,centering]{geometry}

%%%%%%%%%%%%%%%%
% additional packages
\usepackage{amsmath}
\usepackage{array}
\usepackage[all]{xy}
\SelectTips{cm}{}
\usepackage[T1]{fontenc}
\usepackage{booktabs,enumitem,multicol}
\usepackage[defaultsups]{newpxtext}
\usepackage[zerostyle=c,straightquotes]{newtxtt}
\usepackage{newpxmath}
\usepackage{color}
\usepackage[hyperfootnotes=false]{hyperref}
\usepackage{pxjahyper}
\usepackage{hologo}
\usepackage{xspace}
\usepackage{makeidx}\makeindex

% common
\usepackage{ptex-manual}
\let\emph=\origemph
\makeatletter
\newlist{simplelist}{description}1
\setlist[simplelist]{%
  itemsep=0pt, listparindent=1zw, itemindent=10pt,
  font=\normalfont\mdseries, leftmargin=2zw,
  before=\advance\@listdepth\@ne,
  after=\advance\@listdepth\m@ne
}
\makeatother

\def\file#1{\textsf{#1}}
\def\code#1{\texttt{#1}}

%%%%%%%%%%%%%%%%
\makeatletter
\setlist{leftmargin=2zw}
\setlist[description]{labelwidth=2zw,labelindent=1zw,topsep=\medskipamount}

\def\>{\ifhmode\hskip\xkanjiskip\fi}

\def\tsp{_{\mbox{\fontsize\sf@size\z@\ttfamily \char32}}}
\def\tpar{_{\mbox{\fontsize\sf@size\z@\ttfamily \string\par}}}
\def\tign{_{\mbox{\fontsize\sf@size\z@\selectfont --}}}

\usepackage{shortvrb}
\MakeShortVerb*{|}
%%%%%%%%%%%%%%%%

% foreign text -> italic
\def\Foreign#1{\textit{#1}}

\def\_{\leavevmode\vrule width .45em height -.2ex depth .3ex\relax}

\frenchspacing
\begin{document}
\catcode`\<=13
\title{\textsf{\textbf{Guide to \pTeX\ for developers unfamiliar with Japanese}}}
\author{Japanese \TeX\ Development Community\null
\thanks{\url{https://texjp.org},\ e-mail: \texttt{issue(at)texjp.org}}}
\date{version p\the\ptexversion.\the\ptexminorversion\ptexrevision, \today}
\maketitle

\pTeX\ and its variants, \upTeX, \epTeX\ and \eupTeX, are all \TeX\ engines
with native Japanese support.
Its output is always a DVI file, which can be processed by several
DVI drivers with Japanese support including \emph{dvips} and \emph{dvipdfmx}.
Formats based on \LaTeX\ is called \pLaTeX\ when running on \pTeX/\epTeX,
and called \upLaTeX\ when running on \upTeX/\eupTeX.

%%% Target readers of this document: 日本語機能でないパッケージ作者に絞る
\section*{Purpose of this document}

This document is written for developers of \TeX/\LaTeX, who aim to
support \pTeX/\pLaTeX\ and its variants \upTeX/\upLaTeX.
Knowledge of the followings are assumed:
\begin{itemize}
  \item Basic knowledge of Western \TeX\ (Knuthian \TeX, \eTeX\ and \pdfTeX),
  \item ... and its programming conventions.
\end{itemize}

%%% 日本語(の組版・文字コード)の知識は要らない。その話はできる限り避ける
Any knowledge of Japanese (characters, encodings, typesetting conventions etc.)
is not assumed; some explanations are provided in this document when needed.
We hope that this document helps authors of packages or classes
to proceed with supporting \pTeX\ family smoothly.

\begin{quotation}
Note: This English guide (\file{\jobname.pdf}) is \emph{not} meant
to be a complete translation of Japanese manual (\file{ptex-manual.pdf}).
For example, this document does not cover issues regarding Japanese typesettings.
If you are interested in typesetting conventions of Japanese text, please also
refer to \file{japanese.pdf} distributed with \Pkg{babel-japanese}.
\end{quotation}

This document is maintained at:
\url{https://github.com/texjporg/ptex-manual/}

\tableofcontents


\newpage


%%%%%
\part{Brief introduction}% 概論

%%% pTeX とその仲間
\section{\pTeX\ and its variants}

The figure below shows the relationship between engines.
\[
\xymatrix@ur{
 \text{\TeX}   \ar[r]\ar[d] & \text{\eTeX}  \ar[r]\ar[d]
   & \text{\fbox{\pdfTeX}} \\
 \text{\pTeX}  \ar[r]\ar[d] & \text{\fbox{\epTeX}} \ar[d] &\\
 \text{\upTeX} \ar[r]       & \text{\fbox{\eupTeX}}       &
}
\]

\pTeX\ is an old Japanese-specific extension of \TeX82,
which aims to support proper typesetting of Japanese text
but only supports a limited character set, JIS~X~0208 (6879 characters).

\upTeX\ is developed as an extension of \pTeX\ to support full Unicode
characters. It also includes modifications and extensions to overcome
the difficulties of \pTeX\ in processing 8-bit Latin characters
due to conflicts with legacy multibyte Japanese encodings.

\epTeX\ and \eupTeX\ are \eTeX\ extensions of \pTeX\ and \upTeX\ respectively.
In the current release, some extensions derived from \pdfTeX\ and \OMEGA\ are
also available.

%%% 際立った pTeX 系列の特徴
\section{Eminent characteristics of \pTeX\ family}

The most important characteristics of \pTeX\ family can be
summarized as follows:
%%% 欧文と和文が別個に存在する・縦組がある
\begin{itemize}
  \item Japanese characters are interpreted and handled completely apart from
    Western characters. If a pair of two or more 8-bit codes in the input
    matches the pattern of Japanese character codes, it is regarded as one
    Japanese character and given a different \.{catcode} (\.{kcatcode}) value.
  \item There are two text directions; horizontal (\Foreign{yoko-gumi}; 横組)
    and vertical (\Foreign{tate-gumi}; 縦組).
    Two directions can be mixed even within a single document.
\end{itemize}

%%% 欧文 TeX との互換性
\section{Compatibility with Western \TeX}\label{compat}

%%% pTeX/upTeX は Knuthian TeX に対してほぼ上位互換
\pTeX/\upTeX\ are almost upward compatible with Knuthian \TeX,
however, they do not pass the TRIP test.
%%% 入力の8bitの扱いが異なる。フォントの8bitはそのまま
The most important difference lies in the handling of 8-bit code inputs;
some 8-bit Latin characters may be subject to the encoding conversion.
There is no difference in handling 8-bit TFM font.

%%% e-pTeX/e-upTeX は e-TeX に対してほぼ上位互換
\epTeX/\eupTeX\ are almost upward compatible with \eTeX,
however, input handling is similar to \pTeX/\upTeX.
It does not pass the e-TRIP test.
%%% だけど e-TeX はもうなく,pdfTeX の DVI モードがあるだけ
That said, please note that ``raw \eTeX'' is unavailable anymore
in \TL and derived distributions;
they provide a command |etex| only as ``DVI mode of \pdfTeX.''
%%% e-pTeX/e-upTeX は pdfTeX の DVI モードに対して上位互換ではない
You should note that
\epTeX/\eupTeX\ are \emph{not} upward compatible with DVI mode of \pdfTeX,
which will be discussed later in Section \ref{dvi-pdftex}.

%%% e-(u)pTeX の話がメイン
There is no advantage to choose \pTeX/\upTeX\ over \epTeX/\eupTeX,
so we focus mainly on \epTeX/\eupTeX.
% [TODO] 既に pTeX/upTeX が暗に e-pTeX/e-upTeX を指す場合もある

%%% LaTeX ムニャムニャ
\section{\LaTeX\ on \pTeX/\upTeX\ --- \pLaTeX/\upLaTeX}

%%% pLaTeX と upLaTeX ムニャムニャ
Format based on \LaTeX\ is called \pLaTeX\ when running on \pTeX,
and called \upLaTeX\ when running on \upTeX.
%%% カーネルが拡張されている
When building the format, |platex.ltx| (\pLaTeX) or
|uplatex.ltx| (\upLaTeX) loads |latex.ltx| first
and adds some additional commands related to the followings:
\begin{itemize}
  \item Selection of Japanese fonts,
  \item Crop marks (called \Foreign{tombow}; トンボ) for printings,
  \item Adjustment for mixing horizontal and vertical texts.
\end{itemize}
%%% author レベルでは LaTeX とほぼ互換,ただし例外あり
For authors, \pLaTeX/\upLaTeX\ are almost upward compatible with
original \LaTeX, except for the followings:
\begin{itemize}
  \item Order of float objects; in \pLaTeX/\upLaTeX,
    <bottom float> is placed above <footnote>.
    That is, the complete order is
    <top float> $\rightarrow$ <body text> $\rightarrow$
    <bottom float> $\rightarrow$ <footnote>.
  % [TODO] 他にもあるか?
\end{itemize}
%%% developer レベルでは pdfTeX 拡張や pLaTeX カーネルでムニャムニャ
For developers, additional care may be needed,
for changes in the kernel macros and/or the engine difference
(Japanese handling, absence of \pdfTeX\ features, etc).
In recent versions of \TL and its derivatives,
the default engines of \pLaTeX\ and \upLaTeX\ are as follows:
\begin{center}
\begin{tabular}{lll}
 Date & \pLaTeX & \upLaTeX \\ \hline
 \TL2010 & \pTeX & --- \\
 \TL2011 & \epTeX & --- \\
 \TL2012--2022, 2023 initial & \epTeX & \eupTeX \\
 Since 2023-06-01 & \eupTeX\ in legacy (see below) & \eupTeX \\
\end{tabular}
\end{center}
The command |platex| started \epTeX\ (not \pTeX) with preloaded format
|platex.fmt| in \TL2022.
Since 2023-06-01, \pLaTeX\ has switched its engine from \epTeX\ to
\eupTeX\ in ``legacy-encoding-compatibility mode.'' It means that
additional primitives of (\eTeXpre)\upTeX\ is available also on \pLaTeX,
but the internal code of Japanese characters is still non-Unicode
to keep the backward compatibility.
Further information can be found in Section \ref{detecting-uptex}.


\newpage


%%%%%
\part{Details}% 各論

%%% 出力フォーマット
\section{Output format --- DVI}

%%% DVI だけ
The output of \pTeX\ family is always a DVI file.
This is in contrast to the mainstream of \pdfTeX\ in the Western \TeX\ world.

In case you are not familiar with DVI output processing,
first we give some general notice on how to get a ``correct'' output
using \LaTeX\ in DVI mode.

\begin{itemize}
  \item The DVI format is, as its name suggests, inherently driver-independent.
    However, some \LaTeX\ packages (\Pkg{graphicx}, \Pkg{color}, \Pkg{hyperref} etc.)
    embed some \.{special} commands into the DVI, which can be interpreted later
    by some specific DVI driver.
    Such a DVI is no longer driver-independent, thus those are called
    driver-dependent packages.
  \item In almost all major \TeX\ distributions (of course including \TL),
    the default DVI driver is set to |dvips|.
    When you choose to process the resulting DVI file with a driver
    other than dvips (e.g. dvipdfmx) after running \LaTeX,
    you need to pass a proper driver option (e.g. |[dvipdfmx]|) to
    all driver-dependent packages.
\end{itemize}

%%% 日本の DVI ドライバの状況
Now, let's move on to the situation in Japan,
which is slightly complicated due to historical reasons
but may also apply to other countries:
\begin{itemize}
  \item There are two major conventions to pass a proper driver option
    to all driver-dependent packages:
    \begin{enumerate}
      \item To give a driver option to each driver-dependent package:
\begin{verbatim}
\documentclass{article}
\usepackage[dvipdfmx]{graphicx}
\usepackage[dvipdfmx]{color}
\end{verbatim}
      \item To have a driver option as global:
\begin{verbatim}
\documentclass[dvipdfmx]{article}
\usepackage{graphicx}
\usepackage{color}
\end{verbatim}
    \end{enumerate}
    The former convention has been used for many years since 1990s
    when the number of driver-dependent packages was limited.
    But in recent years (around 2010--), there are much more
    driver-dependent packages available. Thus
    we (Japanese \TeX\ experts) advise a global driver option
    rather than individual package options for simplicity,
    but not yet fully widespread.\footnote{The fact that
    there had been a mismatch in option names
    (\code{[dvipdfm]} vs. \code{[dvipdfmx]})
    between packages may also have been part of it;
    \Pkg{geometry} did not understand \code{[dvipdfmx]} option until 2018!}
  \item Many people still see driver options as ``optional'';
    they do without driver options unless really needed.
    For example, the convention of having a global driver option
    does no harm even when no driver-dependent package is used, but
    some users choose to omit a driver option to avoid a warning\footnote{%
    Since \LaTeXe~2020-02-02, this warning is effectively gone. This is due
    to preloading of \Pkg{expl3} into the format, and the driver-dependent
    code of \Pkg{expl3} interprets the global driver option.}:
\begin{verbatim}
LaTeX Warning: Unused global option(s):
    [dvipdfmx].
\end{verbatim}
\end{itemize}

\subsection{Extensions of DVI format in \pTeX\ family}

%%% pTeX の DVI は欧文の横組みだけなら普通。和文が入ると特殊,縦組ならIDも変化
The DVI format output by \pTeX\ family is fully compatible with
Knuthian \TeX, as long as the following conditions are met:
\begin{itemize}
  \item No Japanese characters are typeset.
  \item There is no portion of vertical text alignment.
\end{itemize}

However, some additional DVI commands, which are defined in the
standard \cite{dvistd0} but never used in \TeX82, can come out.
\begin{itemize}
  \item |set2| (129): % |put2| (134) seems to be unused
    Used to typeset a Japanese character with 2-byte code
    (both \pTeX\ and \upTeX).
  \item |set3| (130): % |put3| (135) seems to be unused
    Used to typeset a Japanese character with 3-byte code
    (\upTeX\ only).
\end{itemize}
When \pTeX\ is going to typeset a Japanese character into DVI,
it is encoded in JIS, which is always a 2-byte code.
For this purpose, |set2| or |put2| are used.
When \upTeX\ is going to output a Japanese character into DVI,
it is encoded in UTF-32.
If the code is equal to or less than |U+FFFF|,
the lower 16-bit is used with |set2| or |put2|.
If the code is equal to or greater than |U+10000|,
the lower 24-bit is used with |set3| or |put3|.

In addition, \pTeX/\upTeX\ defines one additional DVI command.
\begin{itemize}
  \item |dir| (255):
    Used to change directions of text alignment.
\end{itemize}
The DVI format in the preamble is always set to 2, as with \TeX82.
On the other hand, the DVI ID in the postamble can be special.
Normally it is set to 2, as with \TeX82; however,
when |dir| (255) appears at least once in a single \pTeX/\upTeX\ DVI,
the |post_post| table of postamble contains $\mathrm{ID} = 3$.

\subsection{DVI drivers with Japanese support}

There are some DVI drivers with Japanese support.
The most eminent drivers are \emph{dvips} and \emph{dvipdfmx}.
Nowadays most of casual Japanese users are using \emph{dvipdfmx} as a DVI driver.
On the other hand, users of \emph{dvips} are unignorable, especially those
working in publishing industry.
In recent years, most of major driver-dependent packages support
both two drivers.

\subsubsection{Using \emph{dvipdfmx}}

A DVI file which is output by \pTeX\ can be converted directly to a PDF file
using dvipdfmx.
For Japanese fonts to be used in the output PDF, dvipdfmx refers to
|kanjix.map| generated by the command |updmap|.
You can use the script |kanji-config-updmap| to change font settings;
please refer to its help message or documentation.

\subsubsection{Using \emph{dvips}}

A DVI file which is output by \pTeX\ can be converted to a PostScript file
using dvips.
For Japanese fonts to be used in the output PostScript, dvips refers to
\code{psfonts.map} generated by the command \code{updmap}.
You can use the script |kanji-config-updmap| to change font settings;
please refer to its help message or documentation.

The resulting PostScript file can then be converted to
a PDF file using Ghostscript (ps2pdf) or Adobe Distiller.
When using Ghostscript, a proper setup of Japanese font must be done
before converting PostScript into PDF.
An easy solution for the setup is to run a script |cjk-gs-integrate|
developed by Japanese \TeX\ Development Community.

\section{Programming on \pTeX\ family}

We focus on programming aspects of \pTeX\ and its variants.

%%% レジスタの数
\subsection{Number of registers and marks}

\pTeX\ and \upTeX\ have exactly the same number ($=256$) of registers
(count, dimen, skip, muskip, box, and token) as Knuthian \TeX.
\epTeX\ and \eupTeX\ in extended mode have more registers;
there are 65536, which is twice as many as 32768 of \eTeX.
Similarly \epTeX\ and \eupTeX\ have 65536 mark classes,
which is twice as many as 32768 of \eTeX.

The following code presents an example of detecting the number of
regsiters and mark classes available:
\begin{verbatim}
  \ifx\eTeXversion\undefined
    % Knuthian TeX, pTeX, upTeX:
    %   256 registers, 1 mark
  \else
    \ifx\omathchar\undefined
      % e-TeX, pdfTeX (in extended mode):
      %   32768 registers, 32768 mark classes
    \else
      % e-pTeX, e-upTeX (in extended mode):
      %   65536 registers, 65536 mark classes
    \fi
  \fi
\end{verbatim}
Here a primitive \.{omathchar}, which is derived from \OMEGA, is used
as a marker of a change file \code{fam256.ch}.\footnote{%
There is another \pTeX-derived engine named \pTeX-ng (or Asiatic \pTeX)
\url{https://github.com/clerkma/ptex-ng}; it is based on
\eTeX\ and \upTeX, but currently does not adopt \code{fam256.ch}
so it has the same number of registers and mark classes as \eTeX.}

%%% 数式ファミリの数
\subsection{Number of math families}

In \pTeX\ and \upTeX,
the number of math fonts is restricted to 16,
each of which can contain 256 characters (same as Knuthian \TeX).
In \epTeX\ and \eupTeX, a change file \code{fam256.ch},
which is derived from \OMEGA, extends the upper limit to 256.
As a consequence, \epTeX\ and \eupTeX\ allows 256 math fonts,
each of which can contain 256 characters.\footnote{\OMEGA\ allows
256 math fonts, each of which can contain 65536 characters.}

For \pLaTeX/\upLaTeX\ users to use more than 16 math fonts,
it is necessary to use macros which exploit \OMEGA-derived primitives
such as \.{omathchar}.
Recent (u)\pLaTeX\ (since 2016/11/29) partially supports this,
and the maximum number of math alphabets that can be defined by
|\DeclareMathAlphabet| is extended to 256 (|\e@mathgroup@top|)
without needing any extension package.
However, symbol fonts are restricted to 16 as
|\DeclareMathSymbol| etc still use the standard \.{mathchar} etc.
A simple solution to use more symbol fonts as well as math alphabets
is to load a package \Pkg{mathfam256}\footnote{%
\url{https://www.ctan.org/pkg/mathfam256}} though it's still preliminary.

%%% 拡張プリミティブ
\subsection{Additional primitives and keywords}
% -- as of r62095
% tex -ini: 322 multiletter control sequences
% ptex -ini: 374 multiletter control sequences (= 322 + p:48 + TL:3 + SyncTeX: 1)
% uptex -ini: 381 multiletter control sequences (= 374 + up:7)
% eptex -ini: 388 multiletter control sequences
% eptex -ini -etex: 486 multiletter control sequences (= 374 + e:66 + ep:45 + TL:1)
% euptex -ini: 395 multiletter control sequences
% euptex -ini -etex: 494 multiletter control sequences (= 381 + e:66 + ep:46 + eup:1 + TL:1)
% etex (pdftex) -ini: 477 multiletter control sequences
% etex (pdftex) -ini -etex: 546 multiletter control sequences

Here we provide only complete lists of additional primitives
of \pTeX\ family in alphabetical order.
The features of each primitive can be found in Japanese edition.

% [TODO] 抜けがないか?
% [TODO] アルファベット順に正しく並んでいるか?
% [TODO] 追加されたバージョン情報は正しいか?

\def\New#1{--- New primitive since #1}
\def\NewKey#1{--- New keyword since #1}
\def\NewMoved#1#2{--- Imported from #1, since #2}

%%% pTeX のやつ
\subsubsection{\pTeX\ additions (available in \pTeX, \upTeX, \epTeX, \eupTeX)}
\begin{simplelist}
 \csitem[\.{autospacing}]
 \csitem[\.{autoxspacing}]
 \csitem[\.{disinhibitglue} \New{p3.8.2 (\TL2019)}]
 \csitem[\.{dtou}]
 \csitem[\.{euc}]
 \csitem[\.{ifdbox} \New{p3.2 (\TL2011)}]
 \csitem[\.{ifddir} \New{p3.2 (\TL2011)}]
 \csitem[\.{ifjfont} \New{p3.8.3 (\TL2020)}]
 \csitem[\.{ifmbox} \New{p3.7.1 (\TL2017)}]
 \csitem[\.{ifmdir}]
 \csitem[\.{iftbox}]
 \csitem[\.{iftdir}]
 \csitem[\.{iftfont} \New{p3.8.3 (\TL2020)}]
 \csitem[\.{ifybox}]
 \csitem[\.{ifydir}]
 \csitem[\.{inhibitglue}]
 \csitem[\.{inhibitxspcode}]
 \csitem[\.{jcharwidowpenalty}]
 \csitem[\.{jfam}]
 \csitem[\.{jfont}]
 \csitem[\.{jis}]
 \csitem[\.{kanjiskip}]
 \csitem[\.{kansuji}]
 \csitem[\.{kansujichar}]
 \csitem[\.{kcatcode}]
 \csitem[\.{kuten}]
 \csitem[\.{noautospacing}]
 \csitem[\.{noautoxspacing}]
 \csitem[\.{postbreakpenalty}]
 \csitem[\.{prebreakpenalty}]
 \csitem[\.{ptexfontname} \New{p4.1.0 (\TL2023)}]
 \csitem[\.{ptexlineendmode} \New{p4.0.0 (\TL2022)}]
 \csitem[\.{ptexminorversion} \New{p3.8.0 (\TL2018)}]
 \csitem[\.{ptexrevision} \New{p3.8.0 (\TL2018)}]
 \csitem[\.{ptextracingfonts} \New{p4.1.0 (\TL2023)}]
 \csitem[\.{ptexversion} \New{p3.8.0 (\TL2018)}]
 \csitem[\.{scriptbaselineshiftfactor} \New{p3.7 (\TL2016)}]
 \csitem[\.{scriptscriptbaselineshiftfactor} \New{p3.7 (\TL2016)}]
 \csitem[\.{showmode}]
 \csitem[\.{sjis}]
 \csitem[\.{tate}]
 \csitem[\.{tbaselineshift}]
 \csitem[\.{textbaselineshiftfactor} \New{p3.7 (\TL2016)}]
 \csitem[\.{tfont}]
 \csitem[\.{tojis} \New{p4.1.0 (\TL2023)}]
 \csitem[\.{toucs} \New{p3.10.0 (\TL2022)}]
 \csitem[\.{ucs} \NewMoved{\upTeX}{p3.10.0 (\TL2022)}\footnotemark]
 \csitem[\.{xkanjiskip}]
 \csitem[\.{xspcode}]
 \csitem[\.{ybaselineshift}]
 \csitem[\.{yoko}]
 \csitem[\texttt{H}\index{H=\texttt{H}}]
 \csitem[\texttt{Q}\index{Q=\texttt{Q}}]
 \csitem[\texttt{zh}\index{zh=\texttt{zh}}]
 \csitem[\texttt{zw}\index{zw=\texttt{zw}}]
\end{simplelist}
\footnotetext{The primitive \.{ucs} was part of ``\upTeX\ additions'' until \TL2021.}

%%% upTeX のやつ
\subsubsection{\upTeX\ additions (available in \upTeX, \eupTeX)}
\begin{simplelist}
 \csitem[\.{disablecjktoken}]
 \csitem[\.{enablecjktoken}]
 \csitem[\.{forcecjktoken}]
 \csitem[\.{kchar}]
 \csitem[\.{kchardef}]
 %\csitem[\.{ucs}] % moved to pTeX 3.10.0
 \csitem[\.{uptexrevision} \New{u1.23 (\TL2018)}]
 \csitem[\.{uptexversion} \New{u1.23 (\TL2018)}]
\end{simplelist}

%%% e-pTeX/e-upTeX の pdfTeX 由来
%%% e-pTeX/e-upTeX の Omega 由来
%%% e-pTeX/e-upTeX の XeTeX/LuaTeX 由来
%%% その他の独自拡張
\subsubsection{\epTeX\ additions (available in \epTeX, \eupTeX)}
\begin{simplelist}
 \csitem[\.{currentspacingmode} \New{191112 (\TL2020)}]
 \csitem[\.{currentxspacingmode} \New{191112 (\TL2020)}]
 \csitem[\.{epTeXinputencoding} \New{160201 (\TL2016)}]
 \csitem[\.{epTeXversion} \New{180121 (\TL2018)}]
 \csitem[\.{expanded} \New{180518 (\TL2019)}]
 \csitem[\.{hfi}]
 \csitem[\.{ifincsname} \New{190709 (\TL2020)}]
 \csitem[\.{ifpdfprimitive} \New{150805 (\TL2016)}]
 \csitem[\.{lastnodechar} \New{141108 (\TL2015)}]
 \csitem[\.{lastnodefont} \New{220214 (\TL2022)}]
 \csitem[\.{lastnodesubtype} \New{180226 (\TL2018)}]
 \csitem[\.{odelcode}]
 \csitem[\.{odelimiter}]
 \csitem[\.{omathaccent}]
 \csitem[\.{omathchar}]
 \csitem[\.{omathchardef}]
 \csitem[\.{omathcode}]
 \csitem[\.{oradical}]
 \csitem[\.{pagefistretch}]
 \csitem[\.{pdfcreationdate} \New{130605 (\TL2014)}]
 \csitem[\.{pdfelapsedtime} \New{161114 (\TL2017)}]
 \csitem[\.{pdffiledump} \New{140506 (\TL2015)}]
 \csitem[\.{pdffilemoddate} \New{130605 (\TL2014)}]
 \csitem[\.{pdffilesize} \New{130605 (\TL2014)}]
 \csitem[\.{pdflastxpos}]
 \csitem[\.{pdflastypos}]
 \csitem[\.{pdfmdfivesum} \New{150702 (\TL2016)}]
 \csitem[\.{pdfnormaldeviate} \New{161114 (\TL2017)}]
 \csitem[\.{pdfpageheight}]
 \csitem[\.{pdfpagewidth}]
 \csitem[\.{pdfprimitive} \New{150805 (\TL2016)}]
 \csitem[\.{pdfrandomseed} \New{161114 (\TL2017)}]
 \csitem[\.{pdfresettimer} \New{161114 (\TL2017)}]
 \csitem[\.{pdfsavepos}]
 \csitem[\.{pdfsetrandomseed} \New{161114 (\TL2017)}]
 \csitem[\.{pdfshellescape} \New{141108 (\TL2015)}]
 \csitem[\.{pdfstrcmp}]
 \csitem[\.{pdfuniformdeviate} \New{161114 (\TL2017)}]
 \csitem[\.{readpapersizespecial} \New{180901 (\TL2019)}]
 \csitem[\.{suppresslongerror} \New{211207 (\TL2022)}]
 \csitem[\.{suppressmathparerror} \New{211207 (\TL2022)}]
 \csitem[\.{suppressoutererror} \New{211207 (\TL2022)}]
 \csitem[\.{Uchar} \New{191112 (\TL2020)}]
 \csitem[\.{Ucharcat} \New{191112 (\TL2020)}]
 \csitem[\.{vadjust} \texttt{pre} \NewKey{210701 (\TL2022)}]
 \csitem[\.{vfi}]
 \csitem[\texttt{fi}\index{fi=\texttt{fi}}]
\end{simplelist}

%%% e-upTeX の独自拡張
\subsubsection{\eupTeX\ additions (available in \eupTeX)}
\begin{simplelist}
 \csitem[\.{currentcjktoken} \New{191112 (\TL2020)}]
\end{simplelist}

%%% ほか
\subsubsection{Other cross-engine additions}
% In the standard build of \TL, Sync\TeX\ extension is unavailable in
% Knuthian \TeX; however, it is enabled in \pTeX\ family.
Sync\TeX\ extension (available in \pTeX, \upTeX, \epTeX, \eupTeX):
\begin{simplelist}
 \csitem[\.{synctex}]
\end{simplelist}

\noindent
\TL additions (available in \pTeX, \upTeX, \epTeX, \eupTeX):
\begin{simplelist}
 \csitem[\.{partokencontext} \New{\TL2022}]
 \csitem[\.{partokenname} \New{\TL2022}]
 \csitem[\.{showstream} \New{\TL2022}] % only e-(u)pTeX, not (u)pTeX
 \csitem[\.{special} \texttt{shipout} \NewKey{230214 (\TL2023)}] % only e-(u)pTeX, not (u)pTeX
 \csitem[\.{tracingstacklevels} \New{\TL2021}]
\end{simplelist}

% [TODO] 引数は何で返り値は何か,expandable?

%%% (e-)TeX にあるが (e-)upTeX にないもの
%%% encTeX 拡張など
\subsection{Omitted primitives and unsupported features}

Compared to Knuthian \TeX\ and \eTeX, some primitives and extensions are
omitted due to conflict with Japanese handling.
\begin{itemize}
 \item The enc\TeX\ extension, including the primitives |\mubyte| etc.,
  is unavailable.
 \item The ML\TeX\ extension, such as |\charsubdef|, is not enabled
  by default. It becomes available with the command-line option |-mltex|,
  but not well-tested.
\end{itemize}

%%% コマンドラインオプションの話

% [TODO] Please also refer to ptex.man1.pdf

\subsection{Behavior of Western \TeX\ primitives}

Here we provide some notes on behavior of Knuthian \TeX\ and \eTeX\ primitives
when used within \pTeX\ family.

\subsubsection{Primitives with limitations in handling Japanese}

Each of the following primitives allows only character codes 0--255;
other codes will give an error ``! Bad character code.''
\begin{quote}
 |\catcode|,
 |\sfcode|,
 |\mathcode|,
 |\delcode|,
 |\lccode|,
 |\uccode|.
\end{quote}

Each of the following primivies has |\...char| in its name,
however, the effective values are restricted to 0--255.
\begin{quote}
 |\endlinechar|,
 |\newlinechar|,
 |\escapechar|,
 |\defaulthyphenchar|,
 |\defaultskewchar|.
\end{quote}

\subsubsection{Primitives capable of handling Japanese}

The following primitives are extended to support Japanese characters:
\begin{cslist}
 \csitem[\.{char} <character code>,
   \.{chardef} <control sequence>=<character code>]
  In addition to 0--255, internal codes of Japanese characters
  (see \ref{kanji-internal}) are allowed.
  For putting Japanese characters, a Japanese font
  (see \ref{japanese-fonts}) is chosen.
  Further information can be found in \ref{chardef}.

 \csitem[\.{font}, \.{fontname}, \.{fontdimen}]
  % [TODO]

 \csitem[\.{accent} <character code>=<character>]
  % [TODO]

 \csitem[\.{if} <token$_1$> <token$_2$>, \.{ifcat} <token$_1$> <token$_2$>]
  Japanese character token is also allowed.
  In that case,
  \begin{itemize}
    \item |\if| tests the internal character code of the Japanese character.
    \item |\ifcat| tests the |\kcatcode| of the Japanese character.
  \end{itemize}
\end{cslist}

\begin{dangerous}
\TeX book describes the behavior of |\if| and |\ifcat| as follows;
\begin{quote}
If either token is a control sequence,
\TeX\ considers it to have character code 256 and category code 16,
unless the current equivalent of that control sequence
has been |\let| equal to a non-active character token.
\end{quote}
However, this includes a lie; in the real implementation of \code{tex.web},
a control sequence is considered to have a category code 0.
\end{dangerous}

\subsection{Case study}

Here we provide some code examples
which may be useful for package developers.
% [TODO] 本当は Based on the above knowledge としたいところだが
% 説明が圧倒的に不足しているので天下り的なコード解説…

%%% pTeX かどうかの判定
\subsubsection{Detecting \pTeX}

Since the primitive |\ptexversion| is rather new (added in 2018),
the safer solution for detecting \pTeX\ is
to test if a primitive |\kanjiskip| is defined.
\begin{verbatim}
  \ifx\kanjiskip\undefined
  \else
    % pTeX / upTeX / e-pTeX / e-upTeX
  \fi
\end{verbatim}

%%% upTeX かどうかの判定
\subsubsection{Detecting \upTeX}\label{detecting-uptex}

\upTeX\ is almost upward compatible with \pTeX, however,
there are two major differences:
\begin{enumerate}
  \item Improvements in the \.{kcatcode} business,
    mainly for better handling of Latin-1 characters and CJK tokens.
  \item Unicode as the default internal Japanese encoding
    (see \ref{kanji-internal}),
    for direct use of its huge character set.
\end{enumerate}

The first difference can be detected by checking if
\.{...cjktoken} primitive is defined.
\begin{verbatim}
  \ifx\enablecjktoken\undefined
  \else
    % upTeX/e-upTeX
  \fi
\end{verbatim}
This can be called ``engine detection'' of \upTeX.

The second difference can be detected by checking if
the character \code{0x2121} (fullwidth space in JIS encoding)
is stored as \hex{3000} internally.
\begin{verbatim}
  \ifx\kanjiskip\undefined
  \else
    \ifnum\jis"2121="3000
      % upTeX/e-upTeX with internal Unicode
    \else
      % pTeX/e-pTeX
      % or, upTeX/e-upTeX with internal EUC-JP or Shift-JIS
    \fi
  \fi
\end{verbatim}
This can be called ''encoding detection'' of \upTeX.

Please note that
the format-build setting of \verb+-kanji-internal=(sjis|euc)+ with
\upTeX\ makes it effectively \pTeX\ regarding the character set,
which means that only JIS~X~0208 character set is supported.
This can be called ``legacy-encoding-compatibility mode''
of \upTeX, where the \.{kcatcode} difference remains
but the internal encoding difference disappears.
This method is used in building |platex.fmt| on \eupTeX,
since 2023-06-01. Therefore, to distinguish \upLaTeX\ from \pLaTeX,
``engine detection'' is not enough; you should use ``encoding detection.''

%%% pTeX のバージョン判定
% [TODO] 何が有用?

%%% 大きな定数を定義する話
\subsubsection{Defining large integer constants}
\label{chardef}

According to \cite{topic} (Section 3.3),
\begin{quote}
A control sequence that has been defined with a \.{chardef} command
can also be used as a <number>.
This fact is used in allocation commands such as |\newbox|.
Tokens defined with \.{mathchardef} can also be used this way.
\end{quote}
Here is the list of primitives which can be used for this purpose
in \pTeX\ family:
\begin{simplelist}
 \csitem[\.{chardef} <control sequence>=<character code>]
  Defines a control sequence to be a synonym for
  \.{char} <character code>.

 \csitem[\.{kchardef} <control sequence>=<character code> (for \upTeX/\eupTeX)]
  Defines a control sequence to be a synonym for
  \.{kchar} <character code>.

 \csitem[\.{mathchardef} <control sequence>=<15-bit number>]
  Defines a control sequence to be a synonym for
  \.{mathchar} <15-bit number>.

 \csitem[\.{omathchardef} <control sequence>=<27-bit number> (for \epTeX/\eupTeX)]
  Defines a control sequence to be a synonym for
  \.{omathchar} <27-bit number>.
\end{simplelist}

The first two (\.{chardef} and \.{kchardef}) are usable only when
the integer being defined is in the range of valid character codes,
which is not necessarily continuous (see \ref{kanji-internal}).
The most efficient and convenient way of defining integer constants
is as follows:
\begin{itemize}
 \item 0--255: \.{chardef}
   % "FF = 255
 \item 256--32767: \.{mathchardef}
   % "7FFF = 32767
 \item 32768--134217727: \.{omathchardef} (only for \epTeX/\eupTeX)
   % "7FFFFFF = 134217727
 \item (optional) 256--2147483647: \.{chardef} (only for \upTeX/\eupTeX)
   % "7FFFFFFF = 2147483647 (+1 => ! Number too big.)
\end{itemize}

%%% 指定のコードの和文トークンを得る方法
\subsubsection{Creating a Japanese character token with a specified code}\label{jtoken-tricks}

Short version:
\begin{itemize}
 \item With \epTeX~191112 or later (\TL2020),
   you can use expandable primitives \.{Uchar} and \.{Ucharcat}.
 \item Otherwise, use the ``\.{kansuji} trick''.
\end{itemize}

\paragraph{The ``\.{kansuji} trick''}
This is a modified version of the ``\.{lowercase} trick''
available in \pTeX\ family.

\begin{dangerous}
Short note on the ``\.{lowercase} trick'':
to create a character token with a specified code value between 0--255
with Knuthian \TeX, the ``\.{lowercase} trick'' can be used; for example,
\begin{verbatim}
  \begingroup
  \lccode`\?=\mycount
  \lowercase{\endgroup \def\X{?}}
\end{verbatim}
defines |\X| which expands to a character number |\mycount|
while the \.{catcode} of |?| (\the\catcode`\?) is preserved.
However, the trick cannot be applied to Japanese characters,
since \pTeX\ family does not support \.{lccode} outside 0--255.
\end{dangerous}

\.{kansuji} is an expandable primitive like \.{number} or \.{romannumeral},
and it converts an integer into its corresponding \Foreign{kanji} notation
called \Foreign{kansuji} (漢数字). The important point here is that
the number-\Foreign{kanji} mapping can be altered by \.{kansujichar}.

Example 1: equivalent to |\def\X{あ}| (JIS code \code{0x2422} is ``あ''):
\begin{verbatim}
  \begingroup
    \kansujichar1=\jis"2422 \xdef\X{\kansuji1}
  \endgroup
\end{verbatim}

Example 2: equivalent to |\def\日本{Japan}|.
\begin{verbatim}
  \begingroup
    \kansujichar5=\jis"467C\relax
    \kansujichar6=\jis"4B5C\relax
    \expandafter\gdef\csname\kansuji56\endcsname{Japan}
  \endgroup
\end{verbatim}

Since \.{kansujichar} accepts only Japanese character code,
the ``\.{kansuji} trick'' and the ``\.{lowercase} trick'' should be
used complementarily.

\paragraph{\.{Uchar}, \.{Ucharcat}}
The ``\.{kansuji} trick'' above includes an assignment of \.{kansujichar}
which is unexpandable.
\epTeX~191112 or later (\TL2020) provides expandable primitives
\.{Uchar} and \.{Ucharcat}, which are derived from \hologo{XeTeX}.
Regardless of their names, and unlike \hologo{XeTeX} or \hologo{LuaTeX},
these primitives do \emph{not} necessarily take
a Unicode value as an argument.
These primitives in \epTeX\ and \eupTeX\ take a valid character code
(see \ref{kanji-internal}) based on the internal Japanese encoding.

\begin{simplelist}
 \csitem[\.{Uchar} <character code>]
  Expands to a character token with specified slot <character code>.
  \begin{itemize}
   \item When an 8-bit number (0--255) is given,
     it expands to a Latin character token with category code 12,
     except for a space character (32) which has category code 10.
   \item When a Japanese character code greater than 255 is given,
     it expands to a Japanese character token with its current category code;
     16--18 for \epTeX, 16--19 for \eupTeX.
     % [TODO] 本当は pTeX では和文文字トークン自体は \kcatcode を
     % 持たないが,簡単のためまあいっか.
  \end{itemize}

 \csitem[\.{Ucharcat} <character code> <category code>]
  Expands to a character token with slot <character code> and
  <category code> specified.
  \begin{itemize}
   \item With \epTeX:
     \begin{itemize}
      \item Only 8-bit number (0--255) are allowed for <character code>;
        that is, only Latin characters can be generated.
      \item The values allowed for <category code> are 1--4, 6--8, 10--13.
     \end{itemize}
   \item With \eupTeX:
     \begin{itemize}
      \item When <character code> is between 0--127,
        only Latin characters can be generated.
        Thus, the values allowed for <category code> are
        1--4, 6--8, 10--13.
      \item When <character code> is between 128--255,
        both Latin and Japanese characters can be generated
        depending on the specified <category code>;
        1--4, 6--8, 10--13: Latin character,
        16--19: Japanese character.
      \item When <character code> is greater than 255,
        only Japanese characters can be generated.
        Thus, the values allowed for <category code> are
        16--19.
     \end{itemize}
  \end{itemize}
\end{simplelist}

%%% pdfTeX と違うところ
\subsection{Difference from \pdfTeX\ in DVI mode}\label{dvi-pdftex}

As stated in Section \ref{compat}, \epTeX/\eupTeX\ are
\emph{not} upward compatible with DVI mode of \pdfTeX,
which is available as the |etex| command in \TL.
Here we list some important differences:

%%% pdfTeX の DVI モードにあって e-(u)pTeX にないプリミティブ
First, some \pdfTeX-specific primitives are absent. Examples:
\begin{itemize}
 \item All primitives specific to PDF output:
   \.{pdfoutput}, \.{pdfinfo}, \.{pdfobj} etc.\footnote{%
    \epTeX/\eupTeX\ has primitives \.{pdfpagewidth} and \.{pdfpageheight};
    this is just because they were convenient for implementing \.{pdfsavepos},
    and their behavior is somewhat different from that of \pdfTeX.
    Also note that \epTeX/\eupTeX\ does not have \.{pdfhorigin}
    and \.{pdfvorigin}.}
 \item All primitives related to micro-typography:
   \.{pdffontexpand}, \.{pdfprotrudechars}, etc.
 \item Some primitives related to handling of strings:
   \.{pdfescapestring}, \.{pdfescapehex} etc.
\end{itemize}


%%% その他ムニャムニャ

%%% ファイルの文字コードはどうするか
\subsection{Recommendation for file encoding}\label{file-enc}

Due to historical reasons,
multiple encodings are commonly used for Japanese text.
Sometimes user documents and distribution files (classes, packages)
may have different encodings.
Among those, the universal UTF-8 and
three major legacy encodings (ISO-2022-JP, EUC-JP, Shift-JIS)
are accepted as input to \pTeX\ family,
depending on the configuration and runtime options.
To make this possible,
\pTeX\ family does code conversion in input and output.

The details are too complicated, so here we propose
the optimum solution for Japanese file encoding
for package/class developers who aim to support \pTeX\ family:

%%% 全部アスキーにしてしまえ
%%% UTF-8 にしたい場合はムニャムニャ
%%% \epTeXinputencoding を使う話
\begin{itemize}
 \item \underline{If you want to distribute files on CTAN/\TL, please use UTF-8.}
   \par
   UTF-8 files are \emph{almost always} safe enough
   for recent \pTeX/\upTeX\ (2018--),
   and the same files will have no problem when read by Western \TeX.
   To secure this ``\emph{almost always}'' to ``\emph{always}'', please add
   below at the beginning of your individual UTF-8 files:\par\quad
   \verb+\ifx\epTeXinputencoding\undefined \else \epTeXinputencoding utf8 \fi+\par
   it will help \pTeX\ family to read forcibly in UTF-8,
   so it becomes \emph{always} safe for 2016--.
 \item \underline{If you aim to support broader legacy environment of \pTeX\ specifically, ...}
   \par
   Extra care is required for missing features and
   different configurations which can lead to failure of reading UTF-8.
   There is no such thing as a perfect solution; instead,
   you should choose between encoding in ISO-2022-JP
   or writing in ASCII characters.
  \begin{itemize}
   \item Encoding in ISO-2022-JP:\par
    All historical versions of \pTeX\ family can always read
    ISO-2022-JP properly because it's a 7-bit encoding
    safely distinguished from others.
    This is why many old packages/classes widely used in Japan
    are particularly encoded in ISO-2022-JP.\par
    On the other hand, ISO-2022-JP is unsupported in Western \TeX;
    also, CTAN/\TL requires some special handling of uploaded files.
   \item Writing in ASCII characters:\par
    Safe for Western \TeX\ and CTAN/\TL, but often requires lots of
    {\TeX}niques and hard-to-read programmings (e.g. generating
    Japanese tokens as in Section \ref{jtoken-tricks}, or encoding into
    hex dump as in \Pkg{bxjalipsum.sty}, ...)
  \end{itemize}
  ... Annoying? Please forget that legacy environment ;-)
\end{itemize}

\begin{dangerous}
For your information, here is the behavior of the common default
configuration available in the latest \TL distribution (since 2023).
\begin{itemize}
 \def\Opt#1{\textcolor{red}{#1}}
 \item \upTeX\ default: always properly reads \Opt{UTF-8} and ISO-2022-JP.
 \item \pTeX\ default: always properly reads ISO-2022-JP.
   It also properly reads \Opt{UTF-8} almost always\footnote{The
   exception of ``almost always'' comes from a failure of guessing;
   at the expense of properly reading a certain amount of
   Shift-JIS and EUC-JP, there are occasional misreading of UTF-8.},
   and successful results of ``guess-input-enc'' conversion of
   Shift-JIS and EUC-JP.
 \item When the command-line option \verb+-kanji=(sjis|euc)+ is
   specified: \Opt{UTF-8} above is replaced with the given encoding.
\end{itemize}
\end{dangerous}

\begin{dangerous}
Note on older versions:
\begin{itemize}
 \item The ``guess-input-enc'' conversion status above is relatively new:
  \begin{itemize}
   \item In \TL2023, it is available for all platform of \TL.
     Default on for (\eTeXpre)\pTeX, off for (\eTeXpre)\upTeX,
     but also controlled by runtime option \code{-(no-)guess-input-enc}.
   \item In \TL2022 and older, it was limited for Windows only;
     also, default on for all of (\eTeXpre)(u)\pTeX.
     For Unix it was not implemented yet.
  \end{itemize}
 \item In \TL2017 and older,
   the default input encoding of (\eTeXpre)\pTeX\ was
   Shift-JIS for Windows, UTF-8 for Unix.
 \item The primitive \.{epTeXinputencoding} was added to
   \epTeX/\eupTeX\ in \TL2016. Older versions does not have it.
 \item Very old distributions by ASCII Corporation (--2009)
   supported only legacy encodings; UTF-8 was not allowed.
\end{itemize}
\end{dangerous}

%%% e-upTeX の入力解析:euc/sjis を語りたくないので utf8.uptex 前提
\subsection{Input handling}

For simplicity, first we introduce how \upTeX\ handles the input
when all files are UTF-8.
%%% 文字の和文扱い・欧文扱い
%%% 和文カテゴリーコード
%%% \...cjktoken
\begin{enumerate}
 \item An input line is stored into the internal buffer.
   (No effective code conversion here for \upTeX\footnote{To be precise,
   it passes through a code conversion; however, this is an identity conversion
   which has no effect because \upTeX\ defaults to internal Unicode.}.)
 \item The input processor reads the buffer.
   Here Japanese character tokens are distinguished from
   ordinary 8-bit character tokens.
 \item ... [TODO]
\end{enumerate}

%%% ちなみに e-pTeX では:euc/sjis があることにだけ触れる
The situation is more complicated in \pTeX.
As described in Section \ref{file-enc}, it accepts UTF-8 input;
however, \pTeX\ uses a legacy encoding as the internal Japanese encoding
(default: Shift-JIS on Windows, and EUC-JP otherwise).
This means that \pTeX\ does code conversion in input and output.
... [TODO]

%%% e-upTeX の和文トークンの話:和文トークンが混ざっているので注意
\subsection{Japanese tokens}

%%% 文字トークンが和文であるかを判定する方法
%%% (1) \meaning が kanji で始まるかどうか
%%% (2) pTeX: 文字コードが 256 以上かどうか
%%% (3) upTeX: 予め作っておいた \kcatcode 16--19 のトークン各々と \ifcat で比較

%%% 和文組版の厳選トピック
\section{Basic introduction to Japanese typesetting}

%%% 和文をやり過ごすために注意すべきこと
This section does not aim to explain Japanese typesetting completely;
here we provide a minimum requirement for ``getting away'' with Japanese.

%%% 空白・ペナルティ挿入:勝手に入ってくる!
\subsection{Automatic insertion of glue and penalties}

Sometimes \pTeX\ family automatically inserts glue and penalties
between characters.
% [TODO] もう少しだけ詳しく

%%% 和文フォント
\subsection{Japanese fonts}\label{japanese-fonts}

%%% 欧文とは別個
\pTeX\ family can have 3 different ``current'' fonts at the same time;
a Latin font, a Japanese font for horizontal writing (\Foreign{yoko-gumi}),
and a Japanese font for vertical writing (\Foreign{tate-gumi}).
The first one is the same as in the Knuthian \TeX,
which is defined in a standard TFM format.
The latter two are specific to \pTeX\ family, which are defined
in a JFM (Japanese \TeX\ font metric) format.\footnote{%
A JFM is a modified version of the standard TFM.
It can be created by (u)pPLtoTF, and decoded by (u)pTFtoPL.
Please also refer to the man pages of these programs
(\code{ppltotf.man1.pdf} and \code{ptftopl.man1.pdf}).}

While typesetting, \pTeX\ family automatically switches between
these 3 fonts, depending on the character code and the writing direction:
\begin{itemize}
  \item For typesetting Latin characters,
    the current Latin font shown by |\the\font| is selected.
  \item For typesetting Japanese characters,
    the current Japanese font suitable for the current writing direction
    is selected. It is shown by |\the\jfont| for horizontal writing
    and |\the\tfont| for vertical writing.
\end{itemize}

%%% \nullfont しても全部消えない
%%% 「和文 \nullfont」は和文フォントが一回もグローバルに
%%% 設定されていないとき = iniTeX 時しかない
In Knuthian \TeX, the primitive \.{nullfont} refers to an ``empty font''
in which all characters are undefined.
However in \pTeX\ family, this is regarded as a Latin font
and there is no equivalent to ``Japanese \.{nullfont}'' by design.
To elaborate, it is possible \emph{only} when no Japanese font is
set globally, i.e. in ini\TeX\ mode.
Once a valid Japanese font is selected, there is no way of
selecting ``Japanese \.{nullfont}'' to discard all characters.

Moreover, \pTeX\ and friends assume that each Japanese font
(except ``Japanese \.{nullfont}'' in ini\TeX\ mode)
contains all valid Japanese character code.
In other words, all Japanese fonts share the same character set
corresponding to the whole valid Japanese character code range.\label{jfont}

\section{Other strange beasts}

%%% 縦組は諦めよう

% [TODO] どこに書くか困ったので最後に:内部コードのアレな話
%%% 和文文字コードは連続でない
\subsection{Internal Japanese encodings}\label{kanji-internal}

The <character code> is a union of the following two:
\begin{itemize}
 \item Range of numbers between 0--255, and
 \item Numbers allowed for internal code of Japanese characters.
\end{itemize}
The former is the same as Knuthian \TeX, but the latter is a problem.

In \upTeX\ (default internal Unicode mode), the range is very simple:
\[ c \ge 0 \]

However in \pTeX, only limited encodings are available;
Shift-JIS as |sjis| (default for \TL Windows), or
EUC-JP as |euc| (otherwise).
The range can be represented as follows:
\[
  c = 256c_1+c_2 \; (c_i\in C_i)
\]
where
\[
  \begin{cases}
    C_1=C_2=\{\hex{a1},\dots,\hex{fe}\} & (\mathtt{euc}), \\
    C_1=\{\hex{81},\dots,\hex{9f}\}\cup\{\hex{e0},\dots,\hex{fc}\},
    C_2=\{\hex{40},\dots,\hex{7e}\}\cup\{\hex{80},\dots,\hex{fc}\} & (\mathtt{sjis}).
  \end{cases}
\]
Therefore, the overall range of <character code> is \emph{not} continuous.
This is similar for ``legacy-encoding-compatibility mode'' of \upTeX.

%%% 和文文字の内部コードとして有効(is_char_kanji が真)かどうかの判定
%%% → ! Bad character code. エラーを出さないために使える
To check whether an integer is a valid Japanese character code or not,
you can use \.{iffontchar} with \epTeX~190709 or later (\TL2020).
Suppose a count register |\mycount| stores an integer, you can do it as follows:
\begin{verbatim}
  \iffontchar\jfont\mycount
    % \mycount is a valid Japanese character code
  \fi
\end{verbatim}
Here the primitive \.{jfont} is used merely as
a representative non-empty\footnote{This assumption is always safe after
one of the standard \pTeX\ formats (e.g. plain \pTeX, \pLaTeX) is loaded.}
Japanese font containing all valid Japanese character code (see \ref{jfont}).

%%% pTeX がサポートする文字集合は JIS X 0208 のみで,内部コードとして有効な範囲より狭い
%%% → JIS X 0208 に含まれる文字を表す整数値の判定もできる
Note that \pTeX\ (not including \upTeX\ with internal Unicode) does not
support typesetting characters outside JIS~X~0208,
which is a subset of accepted range of <character code> described above.
To check if an integer is in the range of JIS~X~0208,
you can use \.{toucs} with \pTeX~p3.10.0 or later (\TL2022):
\begin{verbatim}
  \ifnum\toucs\mycount>0
    % \mycount is in the range of JIS X 0208
  \fi
\end{verbatim}
The primitive \.{toucs} converts an integer value
from an internal Japanese code to a Unicode.
This conversion is based on JIS-Unicode mapping table,\footnote{Defined in
\code{jisx0208.h} of \code{ptexenc} library.} and returns $-1$
if no mapping is available for the input integer.


\newpage

\begin{thebibliography}{99}
 \bibitem{dvistd0} TUG DVI Standards Working Group,
  \textit{The DVI Driver Standard, Level 0}.\\
  \url{https://ctan.org/pkg/dvistd}
 \bibitem{topic} Victor Eijkhout, \textit{\TeX\ by Topic, A \TeX nician's Reference},
  Addison-Wesley, 1992.\\
  \url{https://www.eijkhout.net/texbytopic/texbytopic.html}
\end{thebibliography}

\newpage
\printindex


\end{document}

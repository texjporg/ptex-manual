%#!make ptex-manual.pdf
\documentclass[a4paper,11pt]{article}
\usepackage[textwidth=42zw,lines=40,truedimen,centering]{geometry}

%%%%%%%%%%%%%%%%
% additional packages
\usepackage{amsmath}
\usepackage{array}\usepackage[all]{xy}
\SelectTips{cm}{}
%\usepackage[dvipdfmx]{graphicx}
\usepackage[T1]{fontenc}
\usepackage{booktabs,enumitem,multicol}
\usepackage[defaultsups]{newpxtext}
\usepackage[zerostyle=c,straightquotes]{newtxtt}
\usepackage{newpxmath}
\usepackage[dvipdfmx,hyperfootnotes=false]{hyperref}
\usepackage{pxjahyper}
\usepackage{hologo}
\usepackage{makeidx}\makeindex

% common
\usepackage{ptex-manual}

\def\code#1{\texttt{#1}}

%%%%%%%%%%%%%%%%
\makeatletter
\setlist{leftmargin=2zw}
\setlist[description]{labelwidth=2zw,labelindent=1zw,topsep=\medskipamount}

\def\>{\ifhmode\hskip\xkanjiskip\fi}

\def\tsp{_{\mbox{\fontsize\sf@size\z@\ttfamily \char32}}}
\def\tpar{_{\mbox{\fontsize\sf@size\z@\ttfamily \string\par}}}
\def\tign{_{\mbox{\fontsize\sf@size\z@\selectfont --}}}

\usepackage{shortvrb}
\MakeShortVerb*{|}
%%%%%%%%%%%%%%%%

% logos
\def\epTeX{$\varepsilon$-\pTeX}\def\eTeX{$\varepsilon$-\TeX}
\def\eupTeX{$\varepsilon$-\upTeX}\def\upTeX{u\pTeX}
\def\pTeX{p\kern-.10em\TeX}

\def\_{\leavevmode\vrule width .45em height -.2ex depth .3ex\relax}

\frenchspacing
\begin{document}
\catcode`\<=13
\title{\emph{\pTeX\ manual}}
\author{Japanese \TeX\ Development Community\null
\thanks{\url{https://texjp.org},\ e-mail: \texttt{issue(at)texjp.org}}}
\date{version p\the\ptexversion.\the\ptexminorversion\ptexrevision, \today}
\maketitle

\pTeX\ is a \TeX\ engine with native Japanese support.
% [TODO] Mention difference between \pTeX, \upTeX, \epTeX\ and \eupTeX.
Its output is always a DVI file, which can be processed by several
DVI drivers with Japanese support including {\em dvips} and {\em dvipdfmx}).

Note: This English edition (ptex-manual-en.pdf) is {\em not} meant
to be a complete translation of Japanese edition (ptex-manual.pdf).
We hope that this document helps authors of packages or classes
to support Japanese-enabled engines.

\tableofcontents

\newpage

%%%%%
\part{Japanese Typesetting with \pTeX\ and additional primitives}

\newpage

%%%%%
\part{Compatibility with original \TeX}

\section{Primitives with limitations in handling Japanese}

Each of the following primitives allows only character codes 0--255;
other codes will give an error ``! Bad character code.''
\begin{quote}
 |\catcode|,
 |\sfcode|,
 |\mathcode|,
 |\delcode|,
 |\mathchardef|,
 |\lccode|,
 |\uccode|.
\end{quote}

Each of the following primivies has |\...char| in its name,
however, the effective values are restricted to 0--255.
\begin{quote}
 |\endlinechar|,
 |\newlinechar|,
 |\escapechar|,\\
 |\defaulthyphenchar|,
 |\defaultskewchar|.
\end{quote}

\section{Primitives capable of handling Japanese}

The following primitives are extended to support Japanese characters:
\begin{cslist}
 \csitem[\.{char} <character code>,
   \.{chardef} <control sequence>=<character code>]
  In addition to 0--255, internal codes of Japanese characters are allowed.
  For putting Japanese characters, a Japanese font is chosen.

 \csitem[\.{font}, \.{fontname}, \.{fontdimen}]
  % [TODO]

 \csitem[\.{accent} <character code>=<character>]
  % [TODO]

 \csitem[\.{if} <token$_1$> <token$_2$>, \.{ifcat} <token$_1$> <token$_2$>]
  Japanese character token is also allowed.
  In that case,
  \begin{itemize}
    \item |\if| tests the internal character code of the Japanese character.
    \item |\ifcat| tests the |\kcatcode| of the Japanese character.
  \end{itemize}
\end{cslist}

\begin{dangerous}
\TeX book describes the behavior of |\if| and |\ifcat| as follows;
\begin{quote}
If either token is a control sequence,
\TeX\ considers it to have character code 256 and category code 16,
unless the current equivalent of that control sequence
has been |\let| equal to a non-active character token.
\end{quote}
However, this includes a lie; in the real implementation of tex.web,
a control sequence is considered to have a category code 0.
\end{dangerous}

\newpage

%%%%%
\part{DVI drivers with Japanese support}

Nowadays most of casual Japanese users are using {\em dvipdfmx} as a DVI driver.
On the other hand, users of {\em dvips} are unignorable, especially those
working in publishing industry.

\section{Using {\em dvipdfmx}}

A DVI file which is output by \pTeX\ can be converted directly to a PDF file
using dvipdfmx.
% [TODO] Mention kanji-config-updmap for font setup

\section{Using {\em dvips}}

A DVI file which is output by \pTeX\ can be converted to a PostScript file
using dvips.
% [TODO] Mention kanji-config-updmap again

The resulting PostScript file can then be converted to
a PDF file using Ghostscript (ps2pdf) or Adobe Distiller.
When using Ghostscript, a proper setup of Japanese font must be done
before converting PostScript into PDF.
An easy solution for the setup is a script ``cjk-gs-integrate''
developed by Japanese \TeX\ Development Community.

\newpage
\printindex


\end{document}
